\documentclass[a4paper,12pt]{article}

% Set margins
\usepackage[hmargin=2.5cm, vmargin=3cm]{geometry}

\frenchspacing

% Language packages
\usepackage[utf8]{inputenc}
\usepackage[T1]{fontenc}
% \usepackage[magyar]{babel}

% AMS
\usepackage{amssymb,amsmath}

% Graphic packages
\usepackage{graphicx}

% Colors
\usepackage{color}
\usepackage[usenames,dvipsnames]{xcolor}

% Enumeration
\usepackage{enumitem}

% Links
\usepackage{hyperref}

\begin{document}

\begin{center}
    \Large \textbf{Fuzzy Rule Approximation}
\end{center}

\section{Introduction}

Inputs, changed by the environment

The behavior engine works iteratively.
We can consider any state variable as an output.

The explanation uses zero-based indexing, because most of the programming languages follow this convention also. It hopefully helps the implementation of the proposed method.

\section{Language Variables}

In the proposed behavior description method we can define names for the measures/amounts.
\begin{itemize}
    \item A name cannot contain spaces.
    \item The names are case sensitive.
    \item It cannot start with number.
    \item It cannot contains dot.
    \item It cannot be a keyword.
\end{itemize}
A language variable is a \{name, value\} pair.
For instance
\begin{verbatim}
high is 30
\end{verbatim}
means that the 30 is a \emph{high} value on a particular dimension, metric, amount or measure.
This expression also can be considered as a \textit{statement}.

\section{Universe of Discourse}

A \textit{universe} is an ordered set of language variables, which can be referred by their name or by their index. The pairs are ordered by the values in ascending order.

Let denote $U$ a universe.
\begin{itemize}
    \item $|U|$ is the number of pairs in the set.
    \item $U[0]$ is the \emph{first} value.
    \item $U[a]$ is the value of language variable $a$.
    \item $\min(U)$ is the minimal value of $U$.
    \item $\max(U)$ is the maximal value of $U$.
\end{itemize}

For instance, let define the universe of distance in the following way.

\begin{verbatim}
In the universe of distance
  0 is close and
  10 is medium and
  50 is far.
\end{verbatim}

In this case
\[
|U| = 3, U[0] = 0, U[1] = 10, U[2] = 50, U[\text{close}] = 0, U[\text{medium}] = 10, U[\text{far}] = 50.
\]

\section{Distances on the universe}

We would like to approximate the distance between an observation and a language variable on a given universe.

Let consider a universe $U$, an observation $x$ and a language variable $a$.
The distance of $x$ can be calculated as
\[
\delta(U, x, a) =
\dfrac{|x - U[a])|}{\max(U) - \min(U)}.
\]
It is easy to see, that the maximal distance in the universe is 1.

For instance, let $x = 20$.
\begin{align*}
&\delta(U, 20, \text{close}) = 0.4, \\
&\delta(U, 20, \text{medium}) = 0.2, \\
&\delta(U, 20, \text{high}) = 0.6. \\
\end{align*}

\section{Rules}

For illustration, let define universes for battery charge level and speed.
\begin{verbatim}
In the universe of charge_level
  0 is low and
  100 is high.

In the universe of speed
  0 is slow and
  30 is fast.
\end{verbatim}

A rule defines the relation between inputs and outputs. Therefore, a rule has two parts:
\begin{itemize}
    \item \textit{antecedent}: the input side of the rule,
    \item \textit{consequent}: the output side of the rule.
\end{itemize}

The consequent always contains only one statement, while the antecedent part can contains arbitrary number of statements joined by the \texttt{and} keyword. From this reason, we prefer to write the consequent part of the statement at the beginning of the rules.

\begin{verbatim}
The speed is slow when
  distance is medium and
  charge_level is low.

The speed is fast when
  charge_level is high.
\end{verbatim}

A rule is a point the space of antecedent and consequent dimension. In this particular example, we are in three dimensional space ($\text{distance} \times \text{battery\_level} \times \text{speed}$). The first rule is a point (can be expressed by values as $(10, 0, 0)$). The second rule is a line (or by considering the bounds for this example a line segment of $(d, 100, 30), d \in [0, 50]$).

\section{Rule Distances}

For the further calculations, we have to define the distance of an observation and a rule. The distance from the rule is only depends on the antecedent part of the rule.
We can calculate the distances of the statements individually on all universes. Let denote $\delta_0, \ldots, \delta_{(k-1)}$ these distances for a rule with $k$ statements.
\[
\textbf{s} = (\delta_0, \ldots, \delta_{(k-1)}) \in [0, 1]^k,
\]
\[
\displaystyle
\rho(\textbf{s}) =
\sqrt{\dfrac{\sum_{i=0}^{k-1} \delta_i^2}{k}}
\in [0, 1].
\]
It is the normalized Euclidean distance in the $k$-dimensional space.

\section{Calculation of the Consequences}

Let denote $c_i$ the consequent value of the rules at index $i$. We would like to calculate the consequent of the rulebase.

Let denote the pairs of consequent values and the corresponding rule distances by $\textbf{r}$:
\[
\textbf{r} = \{(c_0, \rho_0), (c_1, \rho_1), \ldots, (c_{m-1}, \rho_{m-1})\}
\]
The operator $\sigma$ for consequent calculation can be defined as:
\[
\sigma(\textbf{r}) = \dfrac{\sum_{i} w_i \cdot c_i}{\sum_{i} w_i},
\quad w_i = \dfrac{1}{\rho_i^2}.
\]


\section{Rulebases and Surfaces}

Let assume that we have $n$ input dimensions. A rulebase definition results an $n + 1$ dimensional rule surface.

\section{Connection of the Rulebases}

\section{Evaluation and Usage}

\section{Example}

\section{Approximation}

\begin{verbatim}
In universe x
  0 is low and
  1 is high.

In universe y
  0 is low and
  1 is high.

In universe z
  0 is low and
  1 is high.

The z is low when y is low.
The z is low when y is high.
The z is high when x is low.
The z is high when x is high.
\end{verbatim}

There is no $\delta, \rho, \sigma$ operators, which is able to interpolate all of the rule points.

For most of the operators, the set of rules and the rule surface are disjoint.

\end{document}
